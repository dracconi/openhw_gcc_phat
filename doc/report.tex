\documentclass[a4paper,twocolumn]{article}

\title{Accelerating the locating of a sound source}
\date{Summer 2025}

\author{
  Malina, Kacper
  \and
  Styrnol, Blazej
}

\begin{document}
\maketitle

\section{Introduction}

As an attempt to accelerate computation of the sound source location,
we have implemented a combination of two algorithms that are often
used together. First algorithm is
GCC-PHAT, which is used for finding the delay between the same sound
arriving at two different microphones. The second one, hereafter
called TDOA-MLAT, is an algorithm for finding the cooridnates of the
source based on the delays given by GCC-PHAT. Note that both
algorithms could be used in other applications, such as radars or
other tasks that involve locating a source of waves, for example
electromagnetic waves.

Since the IP cores themselves only have AXI4-Stream interfaces and
would not be able to communicate with the outside world by themselves,
additional core was prepared. The module that bridges UART to
AXI4-Stream was prepared as a separate entity, so that it can be
reused.

\section{Golden References}

The main focus was on the implementation of the algorithms, and not on
the reliability of the algorithms themselves. After trial
implementations in Python (see \texttt{py/} in repo), it became
apparent that the algorithms are not perfect and are prone to errors.

Since this project was not concerned with checking the algorithms
themselves, it was not evaluated thoroughly.

It has to be noted, that GCC-PHAT is correct most of the time, but
sometimes (around $10\%$)fails. This problem can probably be fixed
with the extension of the Fourier transform length.

The TDOA-MLAT algorithm reliability depends a lot on the input delays
noise, the accepted error in the result and the microphone array setup
(the distances between them). Depending on these factors, it can
either be correct $>90\%$ times, or fail with $<10\%$ pass-rate.

\section{Implementation}

All logic is clocked with the 100MHz clock available on the devboard,
thus all latencies and IPs are generated for that clock
constraint.

The cores were implemented on the following hardware:
\begin{itemize}
\item Spartan-7 \texttt{XC7S50} on RealDigital's Urbana\cite{urbana}
\end{itemize}

\subsection{GCC-PHAT}

The algorithm was completely implemented with Vitis HLS tooling
available from AMD, based on the paper by Knapp and
Carter\cite{gccphat}. The implementation details are available in in
\texttt{doc/details.org}.

Post-synthesis latency in Vitis is reported to be around $350\mu$s for
the transform length of 1024 and 4 16-bit input channels, which yields
around 190Mbps of theoretical throughput. To test the core at this
data rate, 1Gbps Ethernet core could be utilized, but this was
unfeasible for this project. For regular sound recording, usually
44.1kHz is enough, which means that the window of 1024 samples is
$1024 / 44.1\mathrm{kHz} \approx 23\mathrm{ms}$ long. The core only
needs less than 1/65th of that window for processing. There is thus
relatively much time to do other processing. Another alternative would
be to decrease the clock frequency for energy saving.

The results show that it does use considerable amount of resources,
which the report lists at 26k flip-flops and 22k LUTs, which is around
$30\%$ of the utilization of the Spartan-7 populated on Urbana board
used for development. There also is plenty of BRAM used for FIFOs.

It could be useful to use the external RAM for some buffers,
especially for ones that wait for computation to complete. The initial
attempt was performed with \emph{two, multichannel} FFT LOGICORE IP
cores, but it did not fit on the chip. The current approach uses just
one FFT instance, and now there is a lot of data that waits in fast
BRAM, especially the 4 16-bit input channels (16kbits), and the intermediate
result from the reference channel forward FT.

\subsection{TDOA-MLAT}

This algorithm was also implemented completely with Vitis HLS tooling,
based on paper by Schau and Robinson\cite{tdoamlat}. Details are
available in the aforementioned ORG file.


...

\subsection{UART bridge}

The bridge was implemented by hand in Verilog, with two test benches
in System-Verilog. The design details are in the ORG file as well.

The module was tested with the on-board FTDI 2232HQ chip and regular
\texttt{picocom}. It successfully worked with 3M baud-rate in the
in-situ loop-back configuration. The testing loop-back was realized
using the Data FIFO IP core with AXI4-Stream interface, included with
Vivado.

\begin{thebibliography}{9}
\bibitem{gccphat}
  Knapp, C. H. and Carter, G.C., \emph{The Generalized Correlation
  Method for Estimation of Time Delay.} IEEE Transactions on Acoustics,
  Speech and Signal Processing

\bibitem{tdoamlat}
  Schau, H. and Robinson, A., \emph{Passive source localization
  employing intersecting spherical surfaces from\\
  time-of-arrival differences} IEEE Transactions on Acoustics, Speech, and Signal
  Processing

\bibitem{urbana}
  RealDigital \emph{Urbana: Reference and Schematic}

\bibitem{vitis}
  AMD \emph{Vitis High-Level Synthesis User Guide} UG1399

\bibitem{xfft}
  AMD \emph{Fast Fourier Transform LogiCORE IP Product Guide} PG109

\bibitem{vivado}
  AMD \emph{Vivado Design Suite Tcl Command Reference Guide} UG835
  
\end{thebibliography}

\end{document}
